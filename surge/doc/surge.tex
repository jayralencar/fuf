\include{epsf}
\documentstyle[named,10pt,twocolumn]{article}

\topmargin 0pt
\headheight 0pt
\headsep 0pt
\textheight 9in
\textwidth 6.5in
\evensidemargin 0pt
\oddsidemargin 0pt
\marginparwidth 0pt
\marginparsep 0pt

\newenvironment{zitemize}%
{\parskip=0.25\parskip \begin{itemize} \itemsep=0.25\itemsep \parsep=0pt}%
{\end{itemize} \parskip = 4\parskip}

\newenvironment{zenumerate}%
{\parskip=0.25\parskip \begin{enumerate}\itemsep=0.25\itemsep\parsep=0pt}%
{\end{enumerate} \parskip = 4\parskip}

\newcommand{\m}[3]{\multicolumn{#1}{#2}{#3}}
\renewcommand{\topfraction}{.85}
\renewcommand{\textfraction}{.15}
\renewcommand{\bottomfraction}{.85}

\begin{document}
\title{{\bf An Overview of SURGE: a Reusable\\ Comprehensive Syntactic
Realization Component}\\\vspace{.75cm}{\small Technical Report 96-03}\\
{\small Mathematics and Computer Science Deptartment}\\
{\small  Ben Gurion University in the Negev}} 
\vspace{1.25cm}
\author{{\bf Michael Elhadad}\\
{\small \tt http://www.cs.bgu.ac.il/\~{}elhadad}\\
{\small \tt elhadad@cs.bgu.ac.il}\\
{\small Mathematics and Computer Science Dept.}\\
{\small Ben Gurion University in the Negev}\\
{\small Beer Sheva, 84105 Israel}\\
{\small Fax: +972-7 472-909}
\and {\bf Jacques Robin}\\
{\small \tt http://www.di.ufpe.br/\~{}jr}\\
{\small \tt jr@di.ufpe.br}\\
{\small Departamento de Inform\'atica}\\
{\small Universidade Federal de Pernambuco}\\
{\small Av. Prof. Luiz Freire s/n Cidade Universit\'aria}\\
{\small Recife, PE 50740-540 Brazil}\\
{\small Fax: +55-81 271-4925 (or 4281)}
}

\date{}
\maketitle

\begin{abstract}
  This paper describes {\sc surge}, a syntactic realization front-end for
  natural language generation systems.  By gradually integrating
  complementary aspects of various linguistic theories within the
  computational framework of functional unification, {\sc surge} has
  evolved to be one of the most comprehensive grammars of English for
  language generation available today. It has been successfully re-used in
  a variety of generators, with very different architectures and
  application domains.
\end{abstract}

\section{Introduction} 

This paper is an overview of {\sc surge} (Systemic Unification Realization
Grammar of English) a syntactic realization front-end for natural language
generation systems. Developed over the last seven years\footnote{The research
presented in this paper started out while the authors were doing their PhD. at
Columbia University, New York. We are both indebted to Kathleen McKeown for her
guidance and support during those years.} it embeds one of the most
comprehensive computational grammar of English for generation available to
date. It has been successfully re-used in eight generators, that have little in
common in terms of architecture.  It has also been used for teaching natural
language generation at several academic institutions.

We first define the task of a stand-alone syntactic realization component
within the overall generation process and provide criteria for its
evaluation.  We then briefly survey the computational formalism underlying
the implementation of {\sc surge} as well as the syntactic theories that it
integrates.  We then describe in more details the structure of the grammar
and its coverage. Finally, we review the application domains across which
{\sc surge} has been re-used, compare it with other syntactic realization
front-ends for generation and discuss its current limitations.

\section[Reusable Realization Component for NLG]{Reusable
Realization\\Component for NLG} 

Natural language generation has been traditionally divided into three
successive tasks \cite{reiter94}: (1) content determination, (2) content
organization, and (3) linguistic realization. The goal of a re-usable
realization component is to encapsulate the domain-independent part of this
third task. The input to such component should thus be as high-level as
possible without hindering portability.  Independent efforts to define such an
input \cite{comet} \cite{matthiessen87} \cite{yang-et-al} \cite{meteer-book}
have crystalized around a skeletal, partially lexicalized thematic tree
specifying the semantic roles, open-class lexical items and top-level syntactic
category of each constituents. An example {\sc surge} input with the
corresponding sentence is given in Fig.~\ref{example-input}.

\begin{figure}[t]
\rule[.3cm]{.48\textwidth}{.01in}
\scriptsize
Input Specification ($I_1$): 
\[ \left[ \begin{array}{ll}
   cat & clause\\
   process & \left[ \begin{array}{ll}
             type & composite\\
             relation & possessive\\
             lex & ``hand''
             \end{array} \right] \\
   partic  & \left[ \begin{array}{ll}
             agent & \left[ \begin{array}{ll}
                     cat & pers\_pro\\
                     gender & feminine
                     \end{array} \right] \\
             affected & \fbox{1} \left[ \begin{array}{ll}
                        cat & np\\
                        lex & ``editor''
%                        head & \left[ \begin{array}{ll}
%                               lex & ``editor''
%                               \end{array} \right] \\
                        \end{array} \right] \\
             possessor & \fbox{1}\\
             possessed & \left[ \begin{array}{ll}
                         cat & np\\
                         lex & ``draft''\\
%                         head & \left[ \begin{array}{ll}
%                                lex & ``draft''
%                                \end{array} \right] \\
                         \end{array} \right] \\
             \end{array} \right] \\
\end{array} \right] \]                         

Output Sentence ($S_1$): 
``She hands the draft to the editor''
\caption{An example {\sc surge} I/O}
\label{example-input}
\rule{.48\textwidth}{.01in}
\end{figure}

\paragraph{The Syntactic Realization Subtask} 
The task of the realization component is to map such skeletal tree onto a
natural language sentence. It involves the following sub-tasks:
\begin{zenumerate} 
\item {\em Map thematic structure onto syntactic roles}: {\em e.g.,} {\tt
    agent}, {\tt process}, {\tt possessed} and {\tt possessor} onto {\tt
    subject}, {\tt verb-group}, {\tt direct-object} and {\tt indirect-object}
    (respectively) in $S_1$.

\item {\em Control syntactic paraphrasing and alternations} \cite{levin}:
  {\em e.g.,} adding the {\tt (dative-move yes)} feature to $I_1$ would
  result in the generation of the paraphrase ($S_2$): {\em ``She hands the
    editor the draft''}.

\item {\em Prevent over-generation}: 
{\em e.g.,} fail when adding the same {\tt (dative-move yes)} feature to an
  input similar to $I_1$ except that the {\tt possessed} role is filled by {\tt
  ((cat pers-pro))} (for personal pronoun) to avoid the generation of ($S_3$) *
  {\em ``She hands the editor it''}.

\item {\em Provide defaults for syntactic features}: {\em e.g.,} {\tt
    definite} for the NPs of $S_1$.

\item {\em Propagate agreement features}, providing enough input to the
  morphology module: {\em e.g.,} after the {\tt agent} and {\tt process}
  thematic roles have been mapped to the {\tt subject} and {\tt verb-group}
  syntactic roles (respectively), propagate the default {\tt (person
    third)} feature added to the {\tt subject} filler to the {\tt
    verb-group} filler; without such a propagation the morphology module
  would not be able to inflect the verb {\em ``to hand''} as {\em
    ``hands''} in $S_1$.

\item {\em Select closed-class words}: {\em e.g.,} {\em ``she''}, {\em
    ``the''} and {\em ``to''} in $S_1$.

\item {\em Provide linear precedence constraints} among syntactic
  constituents: {\em e.g.,} {\tt subject} $>$ {\tt verb-group} $>$ {\tt
    indirect-object} $>$ {\tt direct-object} once the default active voice
  has been chosen for $S_1$.

\item {\em Inflect open-class words} (morphological processing): {\em
    e.g.,} the verb {\em ``to hand''} as {\em ``hands''} in $S_1$.

\item {\em Linearize the syntactic tree into a string of inflected words}
following the linear precedence constraints.

\item {\em Perform syntactic inference}: As an example of syntactic
  inference, suppose that $I_1$ is embedded within a clause as follows:
\end{zenumerate}

\[ \left[ \scriptsize \begin{array}{ll}
cat & clause\\
process & \left[ \begin{array}{ll}
          lex & ``require''\\
          \end{array} \right] \\
partic &  \left[ \begin{array}{ll}
          soa & ($I1$)\\
          influenced & \fbox{1}\\
          influence & \left[ \begin{array}{ll}
                      cat & pers\_pro\\
                      number & plural
                      \end{array} \right] \\
             \end{array} \right] \\
\end{array} \right] \]                         

$I_1$ fills the {\tt soa (state-of-affair)} role of ``to require'' and the
{\tt influenced} role is co-referent with the {\tt affected} role of $I_1$.
>From this co-reference constraint and the subcategorization constraint of
``to require'' (provided in the lexicon), {\sc surge} deduces that the {\tt
  affected} role of the subordinated clause should be {\em controlled}
\cite{pollard-sag94} (Chap. 7) by the {\tt influenced} role of the matrix
clause and therefore be mapped onto the {\tt subject} syntactic role. This
inferred constraint in turn triggers the choice of the passive voice for
the subordinated clause, resulting in ($S_4$): {\em ``They required the
  draft to be handed by her to the editor.''}


\paragraph{Evaluation Criteria}

A re-usable syntactic realization component (or grammar for short) for language
generation should thrive to satisfy the following properties:
\begin{zenumerate}
\item {\em Comprehensive coverage}: it should be able to generate a large
  set of valid syntactic forms and to output many alternate syntactic forms
  from a single input.

\item {\em Prevention of over-generation}: it should generate only
  grammatical outputs (the counter-part of the previous criterion).

\item {\em Encapsulation of syntactic knowledge}: it should accept inputs
  that can be prepared by the client program or user with as little
  knowledge of syntax as possible.

\item {\em Regular input}: it should require inputs whose structure is
  well-defined and regular, and thus easily producible by a recursive
  client program.

\item {\em Partially canned input}: it should accept inputs where canned
  phrases can co-exist with individual words \cite{reiter95}, to be
  usable across the granularity gradient from simple template systems to
  fully compositional generation systems.

\item {\em Compact input}: it should provide appropriate defaults for all
  syntactic features, so that the input can contain only the few features
  that have non-default values.

\item {\em Extensibility}: it should be easy to add new syntactic
  constructs, and quickly verify their interaction with the rest of the grammar.

\item {\em Efficiency}: it should generate a sentence from the input
  specification as fast as possible.

\item {\em Robustness}: it should be able to generate grammatical fragments
  from ill-formed or incomplete inputs.

\item {\em User-friendliness}: it should be easy to use for computational
  linguists not familiar with the implementation details and easy to learn for
  users with little knowledge of linguistic theory.

\item {\em Versatility}: it should be usable in a wide variety of  
  system architectures and application domains.
\end{zenumerate}

We come back to these criterion in Section~\ref{related-work} when we compare
{\sc surge} with other syntactic realization components.


\section{The {\sc fuf/surge} package}

{\sc surge} is implemented in the special-purpose programming language {\sc
  fuf} \cite{fuf-manual} \cite{elhadad-phd} and it is distributed as a
package with a {\sc fuf} interpreter.  This interpreter has two components:
\begin{zitemize}
\item The {\em functional unifier} that fleshes out the input skeletal tree
  with syntactic features from the grammar.

\item The {\em linearizer} that inflects each word at the bottom of the
  fleshed out tree and prints them out following the linear precedence
  constraints indicated in the tree.
\end{zitemize}

\paragraph{Underlying Formalism}

{\sc fuf} is an extension of the original functional unification formalism
put forward by Kay \cite{kay79}. It is based on two powerful concepts:
encoding knowledge in recursive sets of attribute value pairs called {\em
  Functional Descriptions} (FD) and uniformly manipulating these FDs
through the operation of unification.

Both the input and the output of a {\sc fuf} program are FDs, while the
program itself is a meta-FD called a {\em Functional Grammar} (FG). An FG
is an FD with disjunctions and control annotations. Control annotations are
used in {\sc fuf} for two distinct purposes:
\begin{zitemize}
\item To control recursion on linguistic constituents: the skeletal tree of
  the input FD is fleshed out in top-down fashion by re-unifying each of its
  sub-constituent with the FG.
 
\item To reduce backtracking when processing disjunctions.
\end{zitemize}

The key extensions of {\sc fuf} over Kay's original formalism are its
wide range of control annotations \cite{elhadad-robin92} and its special
constructs to define subsumption among atomic values \cite{elhadad90}.  The
main advantages of {\sc fuf} over {\sc prolog} for
natural language generation are:  
\begin{zitemize} 
\item Partial knowledge assumption (a consequence of relying neither on
  predicate arity nor on attribute order when encoding information in FDs).

\item Default control flow adapted to generation 
(top-down breadth-first recursion on linguistic constituents) and possibility
  to fine-tune it (indexing on grammatical features and dependency-directed
  backtracking and goal freezing on uninstantiated features).

\item Built-in linearizer and morphology component.
\end{zitemize}


\paragraph{Linguistic Framework} 

{\sc surge} represents our own synthesis, within a single working system
and computational framework, of the descriptive work of several
(non-computational) linguists. We took inspiration principally:
\begin{zitemize}
\item From \cite{halliday94} and \cite{winograd} for the overall
  organization of the grammar and the core of the clause and nominal
  sub-grammars.

\item From \cite{fawcett} and \cite{lyons77} for the semantic aspects of
  the clause.

\item From \cite{pollard-sag94} for the treatment of long-distance
  dependencies.

\item From \cite{quirk-et-al85} for the many linguistic phenomena not
  mentioned in other works, yet encountered in many generation application
  domains.
\end{zitemize}

Since many of these sources belong to the {\em systemic} linguistic school,
{\sc surge} is mostly a functional unification implementation of systemic
grammar rules. In particular, the type of FD that it accepts as input specifies
a ``process'' in the systemic sense: it can be an event, a relation, or state
in addition to a ``process'' in its most common, aspectually restricted
sense. The hierarchy of general process types defining the thematic structure
of a clause (and the associated semantic class of its main verb) in the current
implementation is compact and able to cover many clause structures. Yet, the
argument structure and/or semantics of many English verbs do not fit neatly in
any element of this hierarchy \cite{levin}. To overcome this difficulty, {\sc
surge} also includes {\em lexical processes} inspired by {\em lexicalist}
grammars such as the Meaning-Text Theory \cite{melcuk-pertsov} and HPSG
\cite{pollard-sag94}.

A lexical process is a shallower and less semantic form of input, where the
sub-categorization constraints and the mapping from the thematic roles to
the oblique roles \cite{pollard-sag94} are already specified (instead of being
automatically computed by the grammar as is the case for general process
types). The use of specific lexical processes to complement general process
types is an example of the type of theory integration that we were forced
to carry out during the development of {\sc surge}. In the current state of
linguistic research, with each theory focusing on a small set of linguistic
phenomena, such an heterogeneous approach is the best practical strategy to
provide broad coverage.


\section{Organization and Coverage}

At the top-level, {\sc surge} is organized into sub-grammars, one for each
syntactic category. Each sub-grammar encapsulates the relevant part of the
grammar to access when recursively unifying an input sub-constituent of the
corresponding category. For example, generating the sentence {\em ``James buys
the book''} involves successively accessing the sub-grammars for the clause,
the verb group, the nominal group (twice) and the determiner sequence.  Each
sub-grammar is then divided into a set of {\em systems} (in the systemic
sense), each one encapsulating an orthogonal set of decisions, constraints and
features. The main top-level syntactic categories used in {\sc surge} are:
clause, nominal group (or NP), the determiner sequence, the verb group, the
adjectival phrase and the PP. We now describe the systems and coverage
currently implemented in {\sc surge} for each of these categories.

\subsection{The Clause} 

Following \cite{halliday94}, the {\em thematic} roles accepted by {\sc surge}
in input clause specifications first divide into: {\em nuclear} and {\em
satellite} roles.  Nuclear roles, answer the questions ``who/what was
involved?'' about the situation described by the clause.  They include the {\em
process} itself, generally surfacing as the {\tt verb}\footnote{Or as {\tt
direct-object} in the case of clauses headed by support-verbs \cite{gross84}.}
and its associated {\em participants} surfacing as verb arguments.
%and surfacing as
%one of the standard syntactic roles ({\em i.e.,} {\tt verb}, {\tt
%subject}, {\tt direct-object}, {\tt indirect-object}, {\tt
%subject-complement}, {\tt object-complement}, {\tt by-object} and {\tt
%dative-object}).
Satellite roles (also called {\em adverbials}) answer the questions
``when/where/why/how did it happen?'' and surface as the remaining clause
complements.

Semantically, participants are tightly associated with specific nodes in the
process type hierarchy, while satellites are versatile roles compatible with
virtually any process type. Syntactically, participants can be distinguished
from satellites in that\footnote{While each of these three tests has a number
of exceptions, taken together they are very reliable.}: (1) they surface as
{\tt subject} for at least one syntactic alternation \cite{levin}, (2) they
can neither be moved around in the clause nor (3) omitted from the clause while
preserving its grammaticality. 

Following this sub-division of thematic roles, the clause sub-grammar is
divided into four orthogonal systems:
\begin{zitemize}
\item {\em Transitivity}, which handles mapping of nuclear thematic roles
  onto a default core syntactic structure for main assertive clauses.

\item {\em Voice}, which handles departures from the default core syntactic
  structure triggered by the use of syntactic alternations ({\em e.g.,}
  passive or dative moves).

\item {\em Mood}, which handles departures from the default core syntactic
  structure triggered by variations in terms speech acts ({\em e.g.,}
  interrogative or imperative clause) and syntactic functions ({\em e.g.,}
  matrix {\em vs.} subordinate clause).

\item {\em Adverbial}, which handles mapping of satellite roles onto the
  peripheral syntactic structure.
\end{zitemize}


\paragraph{The Transitivity System}

\begin{figure*} [p]
\rule[.3cm]{\textwidth}{.01in}
\small
\begin{tabular}{|l|l|l|l|ccc|}\hline
\m{4}{|c|}{\bf Process type specialization} & 
\m{3}{|c|}{\bf Example sentence} \\\hline
event & material & Agentive & simple & Michael & squats &\\
& & non-effective & & {\em Agent} & & \\\cline{4-7} 
& & & with range & Michael & takes & a shower\\
& & & & {\em Agent} & & {\em Range}\\\cline{3-7}
& & agentive & creative & Michael & makes & falafels\\
& & effective & & {\em Agent} & & {\em Created}\\\cline{4-7}
& & & dispositive & Michael & eats & falafels\\
& & & & {\em Agent} & & {\em Affected}\\\cline{3-3}\cline{5-7}
& & effective & & Falafels & cook &\\
& & non-agentive & & {\em Affected} & &\\\cline{4-7}
& & & creative & Windows & pop &\\
& & & & {\em Created} & &\\\cline{2-7}
& \m{3}{|l|}{mental} & Francisco & thinks &\\
& \m{3}{|l|}{} & {\em Processor} & &\\\cline{5-7}
& \m{3}{|l|}{} & Francisco & liked & how he won\\
& \m{3}{|l|}{} & {\em Processor} & & {\em Phenomenon}\\\cline{2-7}
& \m{3}{|l|}{verbal} & Michael & speaks &\\
& \m{3}{|l|}{} & {\em Sayer} & &\\\cline{5-7}
& \m{3}{|l|}{} & Michael & talked & to Cathie\\
& \m{3}{|l|}{} & {\em Sayer} & & {\em Addressee}\\\cline{5-7}
& \m{3}{|l|}{} & Michael & said & strange stuff\\
& \m{3}{|l|}{} & {\em Sayer} & & {\em Verbalization}\\\hline
relation & \m{2}{|l|}{ascriptive} & attributive & Michael & is & very busy\\
& \m{2}{|l|}{} & & {\em Carrier} & & {\em Attribute}\\\cline{4-7}
& \m{2}{|l|}{} & equative & The hunter & is & the hunted\\
& \m{2}{|l|}{} & & {\em Identified} & & {\em Identifier}\\\cline{2-3}\cline{5-7}
& \m{2}{|l|}{possessive} & & Francisco & owns & a big boat\\
& \m{2}{|l|}{} & & {\em Possessor} & & {\em Possessed}\\
& \m{2}{|l|}{} & & {\em Identified} & & {\em Identifier}\\\cline{4-7}
& \m{2}{|l|}{} & attributive & Francisco & has & long hair\\
& \m{2}{|l|}{} & & {\em Possessor} & & {\em Possessed}\\
& \m{2}{|l|}{} & & {\em Carrier} & & {\em Attribute}\\\cline{2-7}
& locative & \m{2}{|l|}{natural} & It & rains &\\
& & \m{2}{|l|}{} & $\emptyset$ & &\\\cline{3-7}
& & \m{2}{|l|}{existential} & There & is & a catch\\
& & \m{2}{|l|}{} & $\emptyset$ & & {\em Located}\\\cline{3-7}
& & \m{2}{|l|}{accompaniment} & Michael & is & with his wife\\
& & \m{2}{|l|}{} & {\em Located} & & {\em Accompaniment}\\\cline{3-7}
& & \m{2}{|l|}{temporal} & The end & is & soon\\
& & \m{2}{|l|}{} & {\em Located} & & Time\\\cline{3-7}
& & spatial & attributive & Michael & is & far away\\
& & & & {\em Located} & & {\em Location}\\
& & & & {\em Carrier} & & {\em Attribute}\\\cline{4-7}
& & & equative & The tree & reaches & the roof\\
& & & & {\em Located} & & {\em Location}\\
& & & & {\em Identified} & & {\em Identifier}\\\hline
\end{tabular}\\\\\\
\caption{Hierarchy of simple processes}
\label{simple-proc}
\rule{\textwidth}{.01in}
\end{figure*}

\begin{figure*} [p]
\rule[.3cm]{\textwidth}{.01in}
\small
\begin{tabular}{|l|l|l|l|cccc|}\hline
\m{2}{|l|}{\bf Event Type} & \m{2}{|l|}{\bf Relation Type} & 
\m{4}{|l|}{\bf Example sentence}\\\hline 
agentive & non-effective & ascriptive & attributive & Johnson & became & rich
&\\ 
& & & & {\em Agent} & & &\\
& & & & {\em Carrier} & & {\em Attribute} &\\\cline{4-8}
& & & equative & The Bulls & became & the Champs &\\
& & & & {\em Agent} & & &\\  
& & & & {\em Identified} & & {\em Identifier} &\\\cline{3-8}
& & possessive & attributive & Orlando & picked & Shaquille &\\
& & & & {\em Agent} & & &\\
& & & & {\em Carrier} & & {\em Attribute} &\\\cline{3-3}\cline{5-8}
& & locative & & Seikaly & went & to Miami &\\
& & & & {\em Agent} & & &\\
& & & & {\em Carrier} & & {\em Located} &\\\cline{2-3}\cline{5-8}
& dispositive & ascriptive & & Riley & made & New York & a good team\\
& & & & {\em Agent} & & {\em Affected} &\\
& & & & & & {\em Carrier} & {\em Attribute}\\\cline{5-8}
& & & & Riley & made & New York & a good coach\\
& & & & {\em Agent} & & {\em Affected} &\\
& & & & {\em Carrier} & & & {\em Attribute}\\\cline{4-8}
& & & equative & The Nets & made & Coleman & the richest\\
& & & & {\em Agent} & & {\em Affected} &\\  
& & & & & & {\em Identified} & {\em Identifier}\\\cline{3-8}
& & possessive & attributive & The Nets & gave & Coleman & more money\\
& & & & {\em Agent} & & {\em Affected} &\\
& & & & & & {\em Possessor} & {\em Possessed}\\\cline{3-3}\cline{5-8}
& & locative & & Price & threw & the ball & out of bounds\\
& & & & {\em Agent} & & {\em Affected} &\\
& & & & & & {\em Located} & {\em Location}\\\cline{2-3}\cline{5-8}
& creative & ascriptive & & Peter & brews & his beer & very strong\\
& & & & {\em Agent} & & {\em Created} &\\
& & & & & & {\em Carrier} & {\em Attribute}\\\cline{3-3}\cline{5-8}
& & locative & & Michael & opened & windows & on the screen\\
& & & & {\em Agent} & & {\em Created} &\\
& & & & & & {\em Located} & {\em Location}\\\cline{1-3}\cline{5-8}
non-agentive & dispositive & ascriptive & & The game & turned & wide open &\\ 
& & & & {\em Affected} & & &\\
& & & & {\em Carrier} & & {\em Attribute} &\\\cline{3-3}\cline{5-8}
& & possessive & & Coleman & received & an offer &\\ 
& & & & {\em Affected} & & &\\
& & & & {\em Carrier} & & {\em Attribute} &\\\cline{3-3}\cline{5-8}
& & locative & & Coleman & fell & on the floor &\\
& & & & {\em Affected} & & &\\
& & & & {\em Located} & & {\em Location} &\\\cline{2-2}\cline{5-8}
& creative & & & The windows & popped & on the screen &\\ 
& & & & {\em Created} & & &\\
& & & & {\em Located} & & {\em Location} &\\\hline
\end{tabular}\\\\\\
\caption{Hierarchy of composite processes}
\label{compound-proc}
\rule{\textwidth}{.01in}
\end{figure*}

Following \cite{fawcett}, we distinguish among {\em simple} (either event or
relation) and {\em composite} processes (di-transitive, complex-transitive
\cite{quirk-et-al85} (p.721) and causative constructs). For example, we view
the clause {\em ``She gives it to him''} as the conflation of two simple
processes: (1) a simple event {\em ``She gives it''} and (2) the resulting
possessive relation {\em ``It belongs to him''}.  The description of composite
processes is useful to account for alternations like dative move
\cite{levin} (pp.45-49) when the relation is {\tt possessive, locative}
(pp.49-55), {\tt creation} (pp.55-58), {\tt causative} (pp.25-32) and others.

The subtasks of the transitivity system are: (1) to provide an interface to the
semantic encoding used in the client program, (2) to determine which
combinations of predicates are mergeable as composite processes, (3) to
constrain syntactic alternations and (4) to constrain the syntactic realization
of each participant.  {\sc surge} currently covers 21 simple process types and
15 composite process types, thus accepting 36 different nuclear thematic
structures as input.

Figures \ref{simple-proc} and \ref{compound-proc} summarize the set of
participant configurations implemented in Surge.  The main function of the
grammar in this system is (1) to provide an interface to the semantic
encoding used in the client program, (2) determine which combinations of
predicates are expressible as composite processes, (3) encode the relevant
syntactic alternations and (4) constrain the types of fillers for each
participant and their syntactic realization.


\paragraph{The Mood System}

\begin{figure*} [t]
\rule[.3cm]{\textwidth}{.01in}
\small
\begin{tabular}{|l|l|l|l|}\hline
\m{3}{|l|}{\bf Mood features} & {\bf Example sentence}\\\hline
finite & declarative & & {\em The Knicks won the game}\\\cline{2-4}
& interrogative & yes-no & {\em Did the Knicks win the game?}\\\cline{3-4}
& & wh & {\em Who won?}\\\cline{3-4}
& & wh-partial & {\em Whose shirt did they wash?}\\\cline{2-4}
& bound & nominal & {\em That the Knicks won the game} was
expected\\\cline{3-4} 
& & adverbial & Ewing scored 28 points {\em as the Knicks won the
game}\\\cline{2-4}
& relative & simple & the game {\em that the Knicks won}\\\cline{3-4}
& & embedded & the team {\em against which the Knicks won the game}\\\hline
non-finite & imperative & & {\em Win that game!}\\\cline{2-4}
& participle & present & The Knicks celebrated {\em after winning the
game}\\\cline{3-4} 
& & past & {\em Once the game won}, the Knicks celebrated\\\cline{2-4} 
& infinitive & to & The Knicks are able {\em to win the game}\\\cline{3-4}
& & bare & Ewing helped the Knicks {\em win the game}\\\cline{3-4}
& & for-to & Ewing must dominate {\em for the Knicks to win the
game}\\\cline{2-4} 
& verbless & & {\em Although without Ewing}, the Knicks won the
game\\\cline{1-4}  
\end{tabular}
\caption{Mood hierarchy}
\label{mood}
\rule{\textwidth}{.01in}
\end{figure*}

The subtasks of the mood system are: (1) to provide an interface to the
specification of speech acts and interpersonal constraints in the client
program ({\em e.g.,} imperative mood to express a request to a
subordinate), (2) to account for hypotactic relations among clauses ({\em
  e.g.,} subordination, embedding) and (3) to constrain to use of
abbreviated forms ({\em e.g.,} participle and verbless clauses). The mood
of dependent clauses is often inferred by {\sc surge} from its syntactic
function in the matrix and the head verb of the matrix.  {\sc surge}
currently covers 15 different moods illustrated in Fig.\ref{mood}.

\paragraph{The Adverbial System}

\begin{figure*} [p]
\rule[.3cm]{\textwidth}{.01in}
\centerline{
\epsfysize=.2\textheight
\epsfbox{shell.ps}}
\caption{Nuclearity of clause constituents}
\rule{\textwidth}{.01in}
\label{const-nuclearity}
\end{figure*}

\begin{figure*} [p]
\rule[.3cm]{\textwidth}{.01in}
\small
\begin{tabular}{|l|l|l|l|l|r|}\cline{1-6}
\m{3}{|c|}{\bf Semantic} & \m{1}{|c|}{\bf Syntactic} & 
\m{1}{|c|}{\bf Example sentence} & \\\cline{1-3}\cline{6-6}
\m{1}{|c|}{\bf class} & \m{1}{|c|}{\bf role} & \m{1}{|c|}{\bf feature} &
\m{1}{|c|}{\bf category} & & \\\cline{1-6}   
Locative & Location & & Adverb & Bo kissed her {\bf there}. & 1 \\\cline{4-6}
& & & PP & Bo kissed her {\bf on the cheek}. & 2 \\\cline{4-6}
& & & finite S & Bo kissed her {\bf where she wanted}. & 3 \\\cline{4-6}
& & & verbless S & Bo kept the keys {\bf where convenient}. & 4 \\\cline{2-6}
& Direction & & Adverb & Bo slid {\bf down}. & 5 \\\cline{4-6}
& & & NP & Bo drove {\bf this way}. & 6 \\\cline{4-6}
& & & PP & Bo ran {\bf up the hill}. & 7 \\\cline{2-6}
& Destination & & Adverb & Bo went {\bf home}. & 8 \\\cline{4-6}
& & & PP & Bo sent the letter {\bf to LA}.& 9 \\\cline{4-6}
& & & finite S & She sent Bo back {\bf where he belongs}. & 10 \\\cline{2-6}
& Path & & PP & Bo traveled {\bf via Denver}. & 11 \\\cline{2-6}
& Distance & & NP & Bo came {\bf a long way}. & 12 \\\cline{1-6}
Temporal & Duration & & Adverb & Bo stayed there {\bf forever}. & 13 \\\cline{4-6}
& & & NP & Bo stayed there {\bf a long time}. & 14 \\\cline{1-6}
Process & Manner & & PP & Bo kiss her {\bf with love}. & 15 \\\cline{2-6}
& Means & & Adverb & Bo was treated {\bf surgically}. & 16 \\\cline{4-6}
& & & PP & Bo was treated {\bf by surgery}. & 17 \\\cline{2-6}
& Instrument & polar+ & PP & Bo pushed him {\bf with both hands}. &
18 \\\cline{3-6}  
& & polar- & PP & Bo negotiated {\bf without an agent}. & 19 \\\cline{2-6} 
& Comparison & & finite S & Bo went there, {\bf as he did
yesterday}. & 20 \\\cline{4-6}  
& & & infinitive S & Bo played hard, {\bf as if to send the fans a
message}. & 21 \\\cline{4-6} 
& & & -ing S & Bo played great, {\bf as if peeking on schedule}. &
22 \\\cline{4-6}  
& & & -ed S & Bo played hard, {\bf as if not bothered by his knee}. &
23 \\\cline{4-6}   
& & & verbless S & Bo played great, {\bf as if in great form}. & 24 \\\cline{1-6}
Respect & Matter & & PP & Bo talked {\bf about his contract}. & 25 \\\cline{1-6}
Domain & Score & & Score & New York beat Indiana {\bf 90-87}. & 26 \\\cline{1-6}
\end{tabular}\\\\\\
\caption{Predicate Adjuncts}
\label{pred-adjuncts}
\rule{\textwidth}{.01in}
\end{figure*}

\begin{figure*} [p]
\rule[.3cm]{\textwidth}{.01in}
\small
\begin{tabular}{|l|l|l|l|l|r|c}\cline{1-6}
\m{3}{|c|}{\bf Semantic} & \m{1}{|c|}{\bf Syntactic} & 
\m{1}{|c|}{\bf Example sentence} & \\\cline{1-3}\cline{6-6}
\m{1}{|c|}{\bf class} & \m{1}{|c|}{\bf role} & \m{1}{|c|}{\bf feature} & 
\m{1}{|c|}{\bf category} &
& \\\cline{1-6}    
Locative & Location & & Adverb & {\bf There}, Bo kissed her. & 1 \\\cline{4-6}
& & & PP & {\bf On the platform}, Bo kissed her. & 2 \\\cline{2-6}
& Origin & & PP & {\bf From Kansas City}, Bo called Gwen. & 3 & \\\cline{2-6} 
& Distance & & PP & {\bf For a few miles}, the road is damaged. & 
4 \\\cline{1-6} 
Temporal & Time & & Adverb & {\bf Yesterday}, Bo triumphed. & 5 &
\\\cline{4-6} 
& & & NP & {\bf Last year}, Bo triumphed. & 6 \\\cline{4-6}
& & & PP & {\bf On Monday}, Bo triumphed. & 7 & \\\cline{4-6}
& & & finite S & {\bf As he received the ball}, Bo smiled. & 8 &
\\\cline{4-6} 
& & & -ing S & {\bf After receiving the ball}, Bo smiled. & 9 &
\\\cline{4-6} 
& & & -ed S & {\bf Once injured}, Bo grimaced. & 10 & \\\cline{4-6}
& & & verbless S & {\bf Once on the floor}, Bo grimaced. & 11 \\\cline{2-6}
& Duration & & PP & {\bf For a long time}, Bo stayed here. & 12 \\\cline{4-6}
& & & finite S & {\bf Until Bo recovered}, they waited for him. &
13 \\\cline{4-6}  
& & & -ing S & {\bf Since joining the Raiders}, Bo shines. & 14 \\\cline{4-6}
& & & -ed S & {\bf Until forcibly removed}, they will stay. & 15 \\\cline{4-6}
& & & verbless S & {\bf As long as necessary}, they will stay here. &
16 \\\cline{2-6}   
& Frequency & & Adverb & {\bf Often}, Bo scored twice. & 17 \\\cline{4-6}
& & & NP & {\bf Twice a week}, Bo runs 10 miles. & 18 \\\cline{4-6}
& & & PP & {\bf On Sundays}, Bo run 10 miles. & 19 \\\cline{1-6}
Causative & Reason & & PP & {\bf Because of injury}, Bo did not play. & 20
& \\\cline{4-6}
& & & finite S & {\bf Because he was injured}, Bo did not play. & 21
& \\\cline{2-6}  
& Purpose & & PP & {\bf For enough money}, Bo would play anywhere. & 22
 & \\\cline{4-6}
& & & finite S & {\bf So he would get a raise}, Bo held out. & 23  \\\cline{4-6}
& & & infinitive S & {\bf To gain free-agency}, Bo held out. & 24
 & \\\cline{2-6}
& Behalf & & PP & {\bf For the Raiders,} Bo scored twice. & 24 &
\\\cline{1-6} 
Determinative & Addition & & -ing S & {\bf In addition to trading Bo}, they
waived Mike. & 25 \\\cline{2-6}
& Accomp- & polar+ & PP & {\bf With her}, Bo would go
anywhere. & 26 & \\\cline{3-6}
& -animent & polar- & & {\bf Without her}, Bo wouldn't go anywhere. & 
27 \\\cline{2-6}
& Opposition & & PP & {\bf Against the Knicks}, the Nets are 3-1. & 
28 \\\cline{1-6}
Process & Manner & & Adverb & {\bf Tenderly}, Bo kissed her. & 
29 & \\\cline{2-6}            
& Means & & -ing S & {\bf By acquiring Bo}, they strengthen their
defense. & 30 \\\cline{2-6}
& Comparison & polar+ & PP & {\bf Like Mike}, Bo soared above the rim. &
31 \\\cline{3-6} 
& & polar- & PP & {\bf Unlike Mike}, Bo can bat. & 32 \\\cline{1-6}
\end{tabular}\\\\\\
\caption{Sentence Adjuncts}
\label{sent-adjuncts}
\rule{\textwidth}{.01in}
\end{figure*}

\begin{figure*} [p]
\rule[.3cm]{\textwidth}{.01in}
\small
\begin{tabular}{|l|l|l|l|l|r|c}\cline{1-6}
\m{3}{|c|}{\bf Semantic} & \m{1}{|c|}{\bf Syntactic} & 
\m{1}{|c|}{\bf Example sentence} & \\\cline{1-3}\cline{6-6} 
\m{1}{|c|}{\bf class} & \m{1}{|c|}{\bf role} & \m{1}{|c|}{\bf feature} &
\m{1}{|c|}{\bf category} & & \\\cline{1-6}    
Temporal & Co-Event & habit+ & finite S & {\bf Whenever you hurt}, call me. &
1 \\\cline{3-6}   
Causative & & habit- & -ing S & {\bf With his knees hampering him}, 
Bo's defense is sloppy. & 2 \\\cline{4-6}
Blend & & & ed S & {\bf Injured against the Giants}, Bo didn't play. & 3 
 \\\cline{4-6}  
& & & verbless S & {\bf Unable to play due to injury}, Bo stayed home.
& 4  \\\cline{3-6}  
& & habit+ & verbless S & {\bf Whenever in doubt}, call me. & 5  \\\cline{1-6}
Causative & Reason & & finite S & {\bf Since he was injured}, Bo did not play. &
6  \\\cline{2-6} 
& Result & & finite S & They wouldn't give him a raise, {\bf so Bo held
out}. & 7 \\\cline{4-6} 
& & & infinitive S & They waived Bo, {\bf only to see him flourish
elsewhere}. & 8 \\\cline{1-6}
Conditional & Condition & polar+ & finite S & {\bf If he is fully fit}, Bo will
play. & 9 \\\cline{4-6} 
& & & -ed S & {\bf If well-conditioned}, Bo will play. & 10 \\\cline{4-6}
& & & verbless S & {\bf If in great shape}, Bo will play. &
11 \\\cline{3-6}
& & polar- & finite S & {\bf Unless he is fully fit}, Bo won't play. &
12 \\\cline{4-6}  
& & & -ed S & {\bf Unless well-conditioned}, Bo won't play. &
13 \\\cline{4-6} 
& & & verbless S & {\bf Unless in great shape}, Bo won't play. &
14 \\\cline{2-6}
& Concessive & & finite S & {\bf Even if he is fully fit}, Bo won't play. &
15 \\\cline{4-6}  
& Condition & & -ed S & {\bf Even if well-conditioned}, Bo won't play. &
16 \\\cline{4-6} 
& & & verbless S & {\bf Even if in great shape}, Bo won't play. & 
17 \\\cline{2-6}
& Concession & & PP & {\bf In spite of his injury}, Bo played. & 18 \\\cline{4-6}
& & & finite S & {\bf Although he was injured}, Bo played. & 19 \\\cline{4-6}
& & & -ed S & {\bf Although injured}, Bo played. & 20 \\\cline{4-6}
& & & -ing S & {\bf Although hurting}, Bo played. & 21 \\\cline{4-6}
& & & verbless S & {\bf Although out of shape}, Bo played. & 22 \\\cline{1-6}
Determinative & Contrast & & PP & {\bf As opposed to many others}, 
Bo never held out. & 23 \\\cline{4-6}
& & & finite S & {\bf Whereas Mike keeps shooting}, Bo prefers passing. &
24 \\\cline{2-6}
& Exception & & PP & {\bf Except Laettner}, all Olympian were pros. &
25 \\\cline{4-6} 
& & & -ing S & {\bf Except for blocking shots}, Bo can do it all. &
26 \\\cline{2-6} 
& Inclusion & & PP & Bo scored 30 points, {\bf including 5 three-pointers}. &
27 \\\cline{4-6}
& & & -ing S & Bo can do it all, {\bf including blocking shots}. &
28 \\\cline{4-6}  
& & & verbless S & Bo played everywhere, {\bf including in New York}. & 
29 \\\cline{2-6} 
& Substitution & & PP & {\bf Instead of Mike}, they picked Bo. & 30 \\\cline{4-6}
& & & -ing S & {\bf Rather than drafting Mike}, they picked Bo. &
31 \\\cline{4-6} 
& & & infinitive S & {\bf Rather than shoot}, Bo made the perfect pass. & 32\\
& & & (bare) & &  \\\cline{2-6}   
& Addition & & PP & They picked Bo, {\bf as well as Mike}. & 33 \\\cline{1-6}
Respect & Matter & & PP & {\bf Concerning Bo}, they were wrong. & 
34 \\\cline{2-6}
& Standard & & PP & {\bf For a center}, Bo is very quick. & 35 \\\cline{2-6}
& Perspective & & PP & {\bf As a scorer}, Bo remains prolific. & 36
\\\cline{1-6}           
\end{tabular}\\\\\\
\caption{Disjuncts}
\label{disjuncts}
\rule{\textwidth}{.01in}
\end{figure*}

The main tasks of the adverbial system are: (1) to provide an interface to
the semantic encoding used in the client program, (2) to determine the
relative ordering of adverbials within the clause, (3) to restrict possible
co-occurrences of adverbials, (4) to restrict the syntactic realization of
adverbials and (5) to provide default closed-class words ({\em e.g.,}
prepositions such as ``for'' or subordinating conjunctions such as
``when'').

The adverbial system encodes a more subtly constrained mapping from
thematic structures to syntactic roles than the transitivity system. While
nuclear roles surface as mutually exclusive core syntactic functions ({\em
  e.g,} each clause can only have one subject and one direct object),
satellite roles surface as one of three peripheral syntactic functions,
{\em predicate adjuncts}, {\em sentence adjuncts} and {\em disjuncts}
following \cite{quirk-et-al85} (pp. 504-505), with potentially multiple
instances of each. The adjuncts {\em vs.} disjuncts distinction is purely
syntactic and accounts for general alternations ({\em e.g.,} only adjuncts
can be clefted or appear in alternative questions). The distinction between
predicate and sentence adjunct is semantic: the former modifies only the
{\em process} of the clause, while the latter elaborates on the entire
situation described by the clause.  It has subtle repercussions in terms of
syntactic alternations, possible positions and co-occurrence ({\em e.g.,}
predicate adjuncts cannot be fronted). {\sc surge} reflects this distinction
in input where satellite roles can be either {\tt predicate-modifier} which
get mapped onto predicate adjuncts and {\tt circumstancials} which get
mapped onto either sentence adjuncts or disjuncts (depending on semantic
label, syntactic category and lexical content).  The relative distance of
different types of complements from the process is illustrated in
Fig.\ref{const-nuclearity}.

Integrating descriptions by \cite{quirk-et-al85}, \cite{thompson-longacre}
\cite{halliday94}, {\sc surge} can currently map 13 predicate modifiers
onto 26 syntactic realizations of predicate adjuncts and 34 circumstancials
onto 32 syntactic realizations of sentence adjuncts and 36 syntactic
realizations of disjuncts.  Figures \ref{pred-adjuncts},
\ref{sent-adjuncts} and \ref{disjuncts} list the types of predicate
adjuncts, sentential adjuncts and disjuncts are available in {\sc surge}.


\subsection{Nominals} 

\begin{figure*} [t]
\rule[.3cm]{\textwidth}{.01in}
\begin{tabular}{|l|l|l|}\hline
{\bf Grammatical} & {\bf Syntactic} & {\bf Example NP}\\
{\bf function}    & {\bf category}  &\\\hline
Pre-Determiner & - & {\bf all of} his first ten points\\\hline
Determiner & article & {\bf a} victory\\\cline{2-3}
& demonstrative Pronoun & {\bf this} victory\\\cline{2-3}
& question Pronoun & {\bf what} victory?\\\cline{2-3}
& possessive Pronoun & {\bf their} victory\\\cline{2-3}
& NP & {\bf New York's} victory\\\hline
Ordinal & Simple Numeral & the {\bf third} victory\\\cline{2-3}
& Numeral Phrase & their {\bf third straight} victory\\\cline{2-3}
& Discontinuous Numeral Phrase & their {\bf third} victory {\bf in a row}\\\hline
Cardinal & simple Numeral & {\bf seven} victories\\\hline
Quantifier & - & Twice as {\bf many} points\\\hline
Describer & Adjective & an {\bf easy} victory\\\cline{2-3}
& Present Participle & a {\bf smashing} victory\\\cline{2-3}
& Past Participle & a {\bf hard fought} victory\\\hline
Classifier & Noun & a {\bf road} victory\\\cline{2-3}
& Adjective & a {\bf presidential} victory\\\cline{2-3}
& Present Participle & a {\bf winning} streak\\\cline{2-3}
& Noun compound & a {\bf franchise record} victory\\\cline{2-3}
& Measure & a {\bf 33 point} performance\\\hline
Head & common Noun & a {\bf victory}\\\cline{2-3}
& Proper Noun & the hapless {\bf Denver Nuggets} who lost again\\\cline{2-3}
& Noun compound & his league {\bf season high}\\\cline{2-3}
& Measure & a career high {\bf 33 points}\\\cline{2-3}
& Partitive & a season best {\bf 12 of 16 shots}\\\hline
Qualifier & PP & a victory {\bf over the Nets}\\\cline{2-3}
& Relative S & a victory {\bf that will be remembered}\\\cline{2-3}
& To-infinitive S & a victory {\bf to be remembered}\\\cline{2-3}
& For-to-infinitive S & a victory {\bf for them to grab}\\\cline{2-3}
& Present-participle S & a victory {\bf invigorating the Knicks}\\\cline{2-3}  
& Past-participle S & a victory {\bf hard-fought until the end}\\\hline
\end{tabular}
\caption{Syntactic functions inside the NP}
\label{synt-fcts-in-NP}
\rule{\textwidth}{.01in}
\end{figure*}

Nominals are an extremely versatile syntactic category, and except for
limited cases (cf. \cite{vendler}, \cite{levi}, \cite{fries}), no
linguistic semantic classification of nominals has been provided.
Consequently, while for clauses input can be provided in thematic form, for
nominals it must be provided directly in terms of syntactic roles.  The
task of mapping domain-specific thematic relations to the syntactic slots
in an NP is therefore left to the client program ({\em cf.}
\cite{elhadad-mt} for a discussion of this process).

\paragraph{Nominal Types}

{\sc surge} distinguishes among the following nominal types: {\em common
  nouns, proper nouns, pronouns, measures, noun-compounds} and {\em
  partitives}.  Noun-compounds can have a deep embedded structure.
Measures have a very specific syntactic behavior (when used as classifiers
the head noun of the measure does not take a number inflection, {\em e.g.,
  ``a 3 meter boat'' vs. ``the boat measures 3 meters''}).

\paragraph{NP Structure and Functions}

The specific syntactic roles accepted by {\sc surge} for nominal
description are (given in their partial order of precedence):\\ {\em
  determiner-sequence, describer, classifier, head}, and {\em qualifier}.

There are two types of pre-modifiers: describers, which can also appear
as subject complements ({\em e.g., ``a blue book''} = {\em ``the book is
  blue}) and classifiers which cannot \cite{winograd}, since they are an
abbreviated form of an indirect relation \cite{levi} ({\em e.g., ``a theory
  book''} $\neq$ {\em ? ``the book is theory''} but rather {\em ``a theory
  book''} = {\em ``the book is \underline{about} theory''}). {\sc surge}
currently covers a total of 28 syntactic realizations for the 5 functions
above ({\em cf.} \cite{robin-phd} for examples of each), illustrated in 
Fig.~\ref{synt-fcts-in-NP}.


\paragraph{Determiner Sequence}

\begin{figure*} [t]
\rule[.3cm]{\textwidth}{.01in}
\small
\begin{tabular}{|l|l|l|l|}\hline
{\bf Definite} & yes & no & \\ \hline
{\bf Countable} & yes & no & \\ \hline
{\bf Partitive} & yes & no & \\ \hline
{\bf Interrogative} & yes & no &\\ \hline
{\bf Possessive} & yes & no & \\ \hline
{\bf Number} & singular & dual & plural\\ \hline 
{\bf Reference-number} & singular & dual & plural\\ \hline
{\bf Distance} & far & near & \\ \hline
{\bf Selective} & yes & no & \\ \hline
{\bf Total} & + & - & no\\ \hline
{\bf Exact} & yes & no & \\ \hline
{\bf Orientation} & + & - & none\\ \hline
{\bf Degree} & + & - & none\\ \hline
{\bf Evaluative} & yes & no & \\ \hline
{\bf Evaluation} & + & - & \\ \hline
{\bf Status} & none & same & different...\\ \hline
{\bf Case} & objective & subjective & genitive\\ \hline
{\bf Head-cat} & pronoun & common & proper\\ \hline
{\bf denotation-class} & quantity & season & institution...\\ \hline
{\bf Comparative} & yes & no & \\ \hline
{\bf Superlative} & yes & no & \\ \hline
{\bf Cardinal} & number & & \\ \hline
{\bf Ordinal} & number & last & next (+) or previous (-)\\ \hline
{\bf Fraction} & numerator & denominator & \\ \hline
\end{tabular}
\caption{Features controlling the determiner sequence}
\label{detseq}
\rule{\textwidth}{.01in}
\end{figure*}

The determiner-sequence is itself decomposed into the following elements:\\ 
{\tt pre-determiner, determiner, ordinal, cardinal, quantifier}.  This
sub-grammar is special in that it is mainly a closed system, {\em i.e.,}
all lexical differences must be determined by some configuration of
syntactic features.  The determiner sequence sub-grammar currently covers
24 features controlling 109 decision points, illustrated in
Fig.\ref{detseq}. 

\subsection{The Verb Group}

The verb group grammar decomposes in three major systems: {\em tense}, {\em
  polarity} and {\em modality}. {\sc surge} implements the full 36 English
tenses identified in \cite{halliday94} pp.198-207 It provides an interface
to the client program is in terms Allen's temporal relations \cite{allen83}
({\em e.g.,} to describe a past event, the client provides the feature {\tt
  (tpattern (:et :before :st))}, specifying that the event time (et)
precedes the speech time (st)).


\section{Implemented Applications}

{\sc surge} has already been successfully re-used by at least\footnote{It
  is also currently used by several other systems not mentioned here either
  because they are still in preliminary stages or because we have little
  information about them .} eight generators.  The practical usability and
versatility of {\sc surge} is demonstrated by the fact that these systems
differ widely in their application domains, their research emphasis and the
strategy they use to generate the skeletal thematic trees that they pass on
to {\sc surge}.

{\sc cook} \cite{smadja-ci} generates stock market reports from
  semantic messages summarizing the daily fluctuations of several financial
  indexes. It focuses on the semi-automatic acquisition of a declarative
  lexicon including collocations and idioms statistically compiled from a
  large textual corpus. It composes a {\sc surge} input by using {\sc fuf}
  to unify this declarative lexicon with the input semantic message. It
  relies on a fixed, bottom-up control strategy driven by output syntactic
  function: first choose the verb arguments, then the verb and finally the
  adjuncts.

  {\sc advisor-II} \cite{elhadad-phd} generates query responses in an
  interactive student advising system. It focuses on presenting the same
  underlying content as argument towards different conclusions and on
  expressing the same piece of content across linguistic ranks (from
  complex clause down to the determiner sequence). It composes a {\sc
    surge} input by using {\sc fuf} to unify an input semantic network
  (enriched with argumentative constraints) with a declarative, word-based
  lexicon. It relies on a complex, input driven control strategy involving
  goal freezing and dependency-directed backtracking.

{\sc comet} \cite{mckeown-et-al93} generates natural language
  explanations coordinated with graphics in an multi-media tutorial and
  troubleshooting system for a military radio. It focuses on combining an
  wide variety of constraints on lexical choice, including the accompanying
  illustration, the user's vocabulary and previous discourse in addition to
  the usual syntax and domain semantics. It composes a {\sc surge} input by
  using {\sc fuf} to unify text plan fragments with a word-based lexicon.
  It relies on a co-routine control strategy allowing the lexicon to
  request partial re-planning of textual fragments when it reaches an
  impasse while combining constraints.

{\sc streak} \cite{robin-phd} generates newswire style summaries of
  basketball games from game statistics and related historical data. It
  focuses on usage of concise linguistic forms, generation of very complex
  sentences, high paraphrasing power and empirical evaluation of
  architecture and knowledge structures based on corpus analysis. It
  incrementally composes the {\sc surge} input using a revision-based
  control strategy.  It first composes a draft of this input by using {\sc
  fuf} to unify a semantic network of obligatory facts with a phrase
  planner and then a lexicon. This draft is then incrementally revised (at
  times non-monotonically) to opportunistically incorporate as many
  optional or background fact that can fit under corpus-observed space and
  linguistic complexity limits.

{\sc plandoc} \cite{plandoc-walker} generates documentation of telephone
  network extension and upgrade proposals by planning engineers, from the
  trace of the simulation system that they use to come-up with their
  proposals.  Its focuses on high paraphrasing power and on aggregation of
  related semantic messages into complex clauses. It composes {\sc surge}
  inputs using a strategy similar to {\sc advisor-II}, but using
  constraints derived from the extended discourse instead of argumentative
  orientation.

{\sc flowdoc} \cite{flowdoc} generates executive summaries of workflow diagrams
  acquired and displayed using the {\sc showbiz} \cite{showbiz} graphical
  business re-engineering tool. It focuses on pointing out characteristics of
  the workflow relevant to re-engineering but obscured by the diagram
  complexity. It composes {\sc surge} inputs using a strategy similar to {\sc
  streak}'s initial draft building stage, except for the combination
  of compositional generation from a word-based lexicon of general workflow
  terms together with canned phrases for task-specific operations (entered
  by the user during workflow acquisition).

{\sc knight} \cite{lester-phd} generates paragraph-long explanations of
  scientific processes from a large knowledge base in biology. It focuses on
  the robustness of the generator when used in a vast domain and the evaluation
  of the produced output. It composes {\sc surge} inputs procedurally by
  filling FD templates using information extracted from the knowledge base.

Abella's system \cite{abella-phd} generates visual scene descriptions from
camera input in two domains: map-aided navigation and kidney X-rays
analysis. Its strategy to compose {\sc surge} inputs is entirely geared toward
its very specific research focus: validating a fuzzy lexical semantic model of
English spatial prepositions.  It thus first performs visual reasoning to pick
the prepositions best describing the main spatial relations among the input
picture landmarks and then procedurally fills locative clause FD templates
associated with each chosen preposition.

% *** BOY: Apres un talk avec Shimei, j'ai appris que Kathy et al ont encore
% rien ecrit sur Magic et ce qui tourne c'est encore la copie de PLANdoc que
% j'avais pondu en 15j avant de % partir, pourquoi James est fort? Alors j'ai
% commenter out (saves space) 
%
% {\sc magic} \cite{magic} generates multimedia summaries of ICU patient
% condition coordinating written text, speech and graphics.  It focuses on
% producing summaries tailored to various types of users ({\em e.g.,}
% specialist, intern, nurse, administrator}) and on the complex temporal and
% spatial coordination of the various media. 


\section{Related Work} 
\label{related-work}

Three other available systems provide reusable surface realization components:
{\sc mumble} \cite{mumble86}, {\sc nigel} \cite{matthiessen87} and the systems
developed at Cogentex (e.g., \cite{iordanskaja-et-al}). These three systems
differ from {\sc surge} in that they are all developed within a single
linguistic theory (TAGs for {\sc mumble}, systemic functional for {\sc nigel}
and Meaning-Text Theory (MTT) \cite{melcuk-pertsov} for Cogentex's systems),
whereas {\sc surge} integrates several ones. Each system also puts the emphasis
on different dimensions resulting in different strengths and weaknesses.

{\sc mumble}'s determinism (motivated on cognitive grounds) makes it very
efficient. Another of its strengths is the regular input provided by
Meteer's text-structure \cite{meteer-book}. {\sc nigel}'s strengths are
comprehensive coverage and encapsulation of syntactic knowledge through the
inquiry mechanism.  However, the procedural implementation of both these
systems make them weak on extensibility and user-friendliness. In addition,
{\sc nigel} also has a tendency to over-generate. This last weakness has
however recently being turned into a strength by the addition of a
post-processor that filters outputs using a statistical language model
\cite{knight-vh}. This post-processor allows {\sc nigel}'s output to nicely
degrades in the face of ill-formed or incomplete input, making it more
robust than other systems.

MTT-based syntactic realization components are strong on extensibility
(declarative {\sc prolog} implementation) and comprehensive coverage. But the
fact that they must work in tandem with an MTT-based lexicon (the Explanatory
Combinatorial Dictionary) make them weak on versatility and encapsulation of
syntactic knowledge. {\sc surge}'s strengths are comprehensive coverage,
encapsulation of syntactic knowledge, the extensibility and user-friendliness
of its purely declarative implementation and the versatility of its compact,
and regular input where individual words and canned phrases can co-exist.  Its
main weakness, inefficiency for complex sentences, is currently being addressed
by the development of a graph-based, re-implementation the {\sc fuf}
interpreter in C to replace the current list-based implementation in LISP. We
expect {\sc surge}'s run times to improve by an order of magnitude with this
new interpreter. A more systematic use of control annotations in the grammar
could also improve run time though less dramatically.


\section{Future Work}

The development of {\sc surge} itself continues, as prompted by the needs
of new applications, and by our better understanding of the respective
tasks of syntactic realization and lexical choice \cite{elhadad-et-al96}.
We are specifically working on (1) integrating a more systematic
implementation of Levin's alternations within the grammar, (2) extending
composite processes to include mental and verbal ones, (3) modifying the
nominal grammar to support nominalizations and some forms of syntactic
alternations (relying on \cite{fries}) and (4) improving the treatment of
obligatory pronominalization and binding.
As it stands, {\sc surge} provides a comprehensive syntactic realization
component, easy to integrate within a wide range of architectures for
complete generation systems.  It is available on the WWW at {\tt
  http://www.cs.bgu.ac.il/surge/}.

\bibliography{master}
\bibliographystyle{named}
\end{document}







